\documentclass{article}
\usepackage{graphicx,tikz}
\usepackage{tabu}
\usepackage{caption}
\usepackage{subcaption}
\usepackage{hyperref}

\newcommand\bred[1]{\textcolor{red}{\textbf{#1}}}
\begin{document}

\title{Benchmark Circuits for IBM's Quantum Computer}
\date{}
\maketitle

\section{Introduction}
IBM's 5 qubit quantum computer \cite{IBMQ} supports gates from the Clifford+T gate library. 
This repository contains some  Clifford+T circuits that have been transformed to be executed on IBM's {\tt qx2} and {\tt qx4}.

\section{Benchmark Circuits}

The following circuits are available in the folder labeled {\tt original}.
\vspace{5mm}

\begin{tabular}{|l|r|r|r|r|c|}
   \hline
   Name & Qubits & Gates & Depth & T-depth & Source \\ \hline  \hline
   {\tt 01.qc} & 5 & 51 & 28 & 9 & \cite{Qbench}  \\  \hline
   
   {\tt 1.qc} & 3 & 17 & 11 & 6 & \cite{Qbench}  \\  \hline
   
    {\tt 3\_17\_b.qc} & 3 & 33 & 23 & 5 & \cite{DBLP:conf/rc/MillerSD14}  \\  \hline
    {\tt 3\_17\_c.qc} & 3 & 35 & 26 & 6 & \cite{DBLP:conf/rc/MillerSD14}  \\  \hline
    {\tt 3\_17\_d.qc} & 3 & 35 & 24 & 4 & \cite{DBLP:conf/rc/MillerSD14}  \\  \hline
    {\tt 3\_17\_e.qc} & 3 & 33 & 21 & 4 & \cite{DBLP:conf/rc/MillerSD14}  \\  \hline
    
    {\tt 17.qc} & 4 & 43 & 30 & 4 & \cite{Qbench}  \\  \hline
    
    {\tt a2x\_c.qc} & 4 & 31 & 22 & 5 & \cite{DBLP:conf/rc/MillerSD14}  \\  \hline
    {\tt a2x\_e.qc} & 4 & 30 & 20 & 4 & \cite{DBLP:conf/rc/MillerSD14}  \\  \hline
    
    {\tt a3x\_c.qc} & 5 & 48 & 37 & 9 & \cite{DBLP:conf/rc/MillerSD14}  \\  \hline
    {\tt a3x\_c.qc} & 5 & 44 & 33 & 8 & \cite{DBLP:conf/rc/MillerSD14}  \\  \hline
    
   {\tt Full\_Adder\_c.qc} & 4 & 20 & 19 & 7 & \cite{DBLP:conf/rc/MillerSD14}  \\  \hline
   {\tt Full\_Adder\_d.qc} & 4 & 22 & 15 & 2 & \cite{DBLP:conf/rc/MillerSD14}  \\  \hline
   {\tt Full\_Adder\_e.qc} & 4 & 21 & 12 & 2 & \cite{DBLP:conf/rc/MillerSD14}  \\  \hline
   
   {\tt Toffoli\_c.qc} & 3 & 17 & 16 & 6 & \cite{DBLP:conf/rc/MillerSD14}  \\  \hline
   {\tt Toffoli\_d.qc} & 3 & 17 & 12 & 3 & \cite{DBLP:conf/rc/MillerSD14}  \\  \hline
   {\tt Toffoli\_e.qc} & 3 & 17 & 12 & 3 & \cite{DBLP:conf/rc/MillerSD14}  \\  \hline
  \end{tabular} 
  \vspace{5mm}
  
  The transformed circuits---to fit the Q5 architecture---are found in the folder labeled {\tt qx2} and {\tt qx4} respectively.
  Different permutations, produce different results.
  Since the computer has 5 available qubits, circuits can be extended to 5 qubits at no cost.
  The names of the circuits are obtained by taken the original name and appending the permutation to it.
  If the output permutation is different than the input permutation, it is appended to the name.
  For example, the circuit {\tt Full\_Adder\_c\_0132\_0123.qc} takes the input in the permutation $(2 3)$, however, the output qubits are not permuted.
  A summary is given below.
  
  \vspace{5mm}
  \begin{tabu}{|l|r|r|r|r|r|}
   \hline
   Name & lines & G - x2 & Dep &  G - x4 & Dep \\ \hline  \hline
  {\tt 01\_01234.qc} & 5 & 149 &  89 & 157 &  92\\  \hline
  {\tt 01\_01342.qc} & 5 & \bred{77} & \bred{38} &  & \\  \hline
  {\tt 01\_10342.qc} & 5 &  &  & \bred{77} &  40 \\  \hline
  \tabucline[2pt]{-}
  
  {\tt 1\_01234.qc} & 5 & 28 & 15  &  24 & 13  \\  \hline
  {\tt 1\_02134.qc} & 5 & \bred{24} &  12 & &  \\  \hline
  {\tt 1\_01234.qc} & 5 & &  & \bred{24}  & 13  \\  \hline
  \tabucline[2pt]{-}
  
   {\tt 3\_17\_b\_01234.qc} & 5 & 49 & 29 & 55 & 33  \\  \hline
   {\tt 3\_17\_b\_02134.qc} & 5 & 43 &  26 & &  \\  \hline
   {\tt 3\_17\_b\_12034.qc} & 5 &  &  & 43 & 24  \\  \hline
   {\tt 3\_17\_c\_01234.qc} & 5 & 49 &  30 & 53 & 34 \\  \hline
   {\tt 3\_17\_c\_02134.qc} & 5 & 43 & 27 & &   \\  \hline
   {\tt 3\_17\_c\_20134.qc} & 5 &  &  & 43 &  27 \\  \hline
   {\tt 3\_17\_d\_01234.qc} & 5 & 49 & 26 & 53 &  29 \\  \hline
   {\tt 3\_17\_d\_02134.qc} & 5 & 45 & 25 & &   \\  \hline
   {\tt 3\_17\_d\_20134.qc} & 5 &  &  & 45 &  25 \\  \hline
   {\tt 3\_17\_e\_01234.qc} & 5 & 47 & 26 & 49 &  26 \\  \hline
   {\tt 3\_17\_e\_02134.qc} & 5 & \bred{41} & \bred{23} & &   \\  \hline
   {\tt 3\_17\_e\_20134.qc} & 5 &  &  & \bred{41} &  \bred{23} \\  \hline
   \tabucline[2pt]{-}
   
    {\tt 17\_01234.qc} & 5 & 141 & 92 & 153 & 97 \\  \hline
    {\tt 17\_20341.qc} & 5 & \bred{101} & \bred{67} & &  \\  \hline
    {\tt 17\_10324.qc} & 5 &  &  & 109 &  68 \\  \hline
    \tabucline[2pt]{-}
    
   {\tt a2x\_c\_01234.qc} & 5 & 85 & 58 & 75 & 45 \\  \hline
   {\tt a2x\_c\_02341.qc} & 5 & 59 & 41 & 59 &  38 \\  \hline
   {\tt a2x\_e\_01234.qc} & 5 & 70 & 44  & 60 & 36  \\  \hline
   {\tt a2x\_e\_02341.qc} & 5 & \bred{52} &  \bred{30} & &  \\  \hline
    {\tt a2x\_e\_31204.qc} & 5 & &   & 39 & 23 \\ 
   \tabucline[2pt]{-}
   
   {\tt a3x\_c\_01234.qc} & 5 & 176 & 127 & 144 & 103 \\  \hline
   {\tt a3x\_c\_10324.qc} & 5 & 86 & 54 & 52 & 39 \\  \hline
   {\tt a3x\_d\_01234.qc} & 5 & 156 & 110 & 134 &  90 \\  \hline
   {\tt a3x\_d\_01324.qc} & 5 & 66 & 38 & &  \\  \hline
    {\tt a3x\_d\_01324.qc} & 5 &  & & \bred{54} &  \bred{37} \\  \hline
    \tabucline[2pt]{-}
    
   {\tt Full\_Adder\_c\_01234.qc} & 5 & 60 &  48 &  60 & 40  \\  \hline
   {\tt Full\_Adder\_c\_01324.qc} & 5 & 28 &  22 &    & \\  \hline
    {\tt Full\_Adder\_c\_32104.qc} & 5 & & &    30  &  25  \\  \hline
    {\tt Full\_Adder\_c\_0132\_0123.qc} & 4 & \bred{24} & & &    \\  \hline
   {\tt Full\_Adder\_d\_01234.qc} & 5 & 74 & 50 & 64 &  38  \\  \hline
   {\tt Full\_Adder\_d\_01324.qc} & 5 & 42 & 25  & &    \\  \hline
   {\tt Full\_Adder\_d\_10324.qc} & 5 &  &  & 32 &  19  \\  \hline
   {\tt Full\_Adder\_e\_01234.qc} & 5 & 53 & 32 & 65 & 36   \\  \hline
   {\tt Full\_Adder\_e\_01324.qc} & 5 & 37 & 23 & &    \\ \hline
    {\tt Full\_Adder\_d\_10324.qc} & 5 &  &  & 31 &  17  \\  
    \tabucline[2pt]{-}
   
   {\tt Toffoli\_c\_01234.qc} & 5 & \bred{17} & 14 & 31 & 23  \\  \hline
   {\tt Toffoli\_c\_21034.qc} & 5 &  &  & \bred{17} & 14 \\  \hline
   {\tt Toffoli\_d\_01234.qc} & 5 & 25 & 16 & 33 & 19 \\  \hline
   {\tt Toffoli\_d\_02134.qc} & 5 &  &  & 25 & 15 \\  \hline
   {\tt Toffoli\_e\_01234.qc} & 5 & 27 & 14 & 29 &  15 \\  \hline
   {\tt Toffoli\_e\_12034.qc} & 5 & 23 & 14 & & \\  \hline
   {\tt Toffoli\_e\_10234.qc} & 5 &  &  & 23 &  14 \\  \hline
\end{tabu} 
 
  
  \vspace{5mm}

The same circuits are available (in IBM format) in the folder labelled {\tt qasm}.
\bibliographystyle{plain}
\bibliography{doc_lit} 

\end{document}