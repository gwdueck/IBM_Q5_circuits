\documentclass{article}
\usepackage{graphicx,tikz}
\usepackage{caption}
\usepackage{subcaption}
\usepackage{hyperref}
\begin{document}

\title{Benchmark Circuits for IBM's Quantum Computer}
\date{}
\maketitle

\section{Introduction}
IBM's 5 qubit quantum computer \cite{IBMQ} supports gates from the Clifford+T gate library. 
This repository contains some  Clifford+T circuits that have been transformed to be executed on IBM's Q5.

\section{Benchmark Circuits}

The following circuits are available in the folder labeled {\tt original}.
\vspace{5mm}

\begin{tabular}{|l|r|r|r|r|c|}
   \hline
   Name & Qubits & Gates & Depth & T-depth & Source \\ \hline  \hline
   {\tt Full\_Adder\_c.qc} & 4 & 20 & 19 & 7 & \cite{DBLP:conf/rc/MillerSD14}  \\  \hline
   {\tt Full\_Adder\_d.qc} & 4 & 22 & 15 & 2 & \cite{DBLP:conf/rc/MillerSD14}  \\  \hline
   {\tt Full\_Adder\_e.qc} & 4 & 21 & 12 & 2 & \cite{DBLP:conf/rc/MillerSD14}  \\  \hline
   {\tt Toffoli\_c.qc} & 3 & 17 & 16 & 6 & \cite{DBLP:conf/rc/MillerSD14}  \\  \hline
   {\tt Toffoli\_d.qc} & 3 & 17 & 12 & 3 & \cite{DBLP:conf/rc/MillerSD14}  \\  \hline
   {\tt Toffoli\_e.qc} & 3 & 17 & 12 & 3 & \cite{DBLP:conf/rc/MillerSD14}  \\  \hline
  \end{tabular} 
  \vspace{5mm}
  
  The transformed circuits---to fit the Q5 architecture---are found in the folder labeled {\tt IBM}.
  Different permutations, produce different results.
  Since the computer has 5 available qubits, circuits can be extended to 5 qubits at no cost.
  The names of the circuits are obtained by taken the original name and appending the permutation to it.
  A summary is given below.
  
  \vspace{5mm}
  \begin{tabular}{|l|r|r|r|r|}
   \hline
   Name & Qubits & Gates & Depth & T-depth  \\ \hline  \hline
   {\tt Full\_Adder\_c\_01234.qc} & 5 & 60 &  &    \\  \hline
   {\tt Full\_Adder\_c\_01324.qc} & 5 & 28 &  &    \\  \hline
  {\tt Full\_Adder\_d\_01234.qc} & 5 & 74 &  &    \\  \hline
   {\tt Full\_Adder\_d\_01324.qc} & 5 & 42 &  &    \\  \hline
     {\tt Full\_Adder\_e\_01234.qc} & 5 & 55 &  &    \\  \hline
   {\tt Full\_Adder\_e\_01324.qc} & 5 & 37 &  &    \\  \hline

  \end{tabular} 
  \vspace{5mm}


\bibliographystyle{plain}
\bibliography{doc_lit} 

\end{document}